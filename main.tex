\documentclass[a4paper, 11pt]{article}

\usepackage[utf8]{inputenc}

\usepackage[T1]{fontenc}

\usepackage[english]{babel}

\usepackage{graphicx}

\usepackage{multicol}

\usepackage{floatrow}

\usepackage[margin = 1in]{geometry}

\usepackage{float}

\usepackage[hidelinks, urlcolor=cyan]{hyperref}

\usepackage{url}

\usepackage{natbib}

\bibliographystyle{abbrvnat}
\setcitestyle{authoryear,open={(},close={)}}

\usepackage{csquotes}

\usepackage{fancyhdr}

%\addbibresource{references.bib}

\title{\Large BINF-402 Project \\
\huge Differential expression analysis of micro-RNA transcriptome between pancreas, prostate and gastrocnemius medialis tissues}


\author{Léopold Guyot}

\date{\today}

\begin{document}

\pagestyle{fancy}
\setlength{\headheight}{32.3pt}
\fancyhead{}\fancyfoot{}
\fancyhead[L]{\includegraphics[width = 0.2\textwidth]{Figures/LOGO_Universite _libre_bruxelles.png}}
\fancyhead[R]{Differencial expression analysis of microRNA transcriptomes}
\fancyfoot[R]{\thepage}

\maketitle

\begin{multicols}{2}
\section{Introduction}
This analysis investigates microRNA expression variations across three distinct tissues; prostate gland, pancreas body, and gastrocnemius medialis. For this a simple workflow will be used, consisting of quality control of the reads, followed by a filtering, then a mapping to finish we a classic differential expression analysis. The goal is to unveil tissue-specific expression patterns of miRNA. Tissue selection is strategic, anticipating closer miRNA expression patterns between prostate and pancreas, both glandular, and unique signatures in gastrocnemius medialis, a muscle tissue.

\section{Methods}
All the data processing was done using the language R \citep{Rlang} and several packages.
\subsection{Data Retrieval \small (\href{https://github.com/leopoldguyot/BINF-402_Transcriptomic_Project/blob/main/retrieve_data.R}{retrieve\_data.R})}
All the data sets used in this project have been retrieve from the ENCODE database \citep{luo2020new}.
Three tissues have been selectionned; pancreas body, prostate gland and the gastrocnemius medialis tissue.
For each tissue, data used was coming from two distinct experiments, each comprising two replicates, thereby totaling four replicates per tissue (cf. \href{https://github.com/leopoldguyot/BINF-402_Transcriptomic_Project/blob/data/sample_table_links.csv}{"data/sample\_table\_links.csv"} for file accession numbers).

The original UCSC hg38 genome was used as reference for the mapping. The NCBI accession for this genome is \href{https://www.ncbi.nlm.nih.gov/datasets/genome/GCF_000001405.26/}{GCA\_000001405.26}.


\subsection{Read Quality Control}
present QC before and after filtering

\begin{figure}[H]
    \centering
    \includegraphics[width=1\columnwidth]{Figures/QC_plots/Cycle_Mean_quality.pdf}
    \caption{Caption of the figure.}
    \label{fig:example}
\end{figure}

\subsection{Mapping}
present mapping stats
\subsection{Differential Expression Analysis}
%\subsection{Gene Ontology}

\section{Results}

Present the overall results => Differential Expression Analysis


\section{Discussion}
???? maybe not include it
=> things I could Improve


\section{Conclusion}


\bibliography{references}

\end{multicols}
\end{document}
